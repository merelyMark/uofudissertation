Visualizing surfaces is a fundamental technique in computer science and is frequently used across a wide range of fields such as computer graphics, biology, engineering, and scientific visualization. In many cases, visualizing an interface between boundaries can provide meaningful analysis or simplification of complex data. Some examples include physical simulation for animation, multimaterial mesh extraction in biophysiology, flow on airfoils in aeronautics, and integral surfaces. However, the quest for high-quality visualization, coupled with increasingly complex data, comes with a high computational cost. Therefore, new techniques are needed to solve surface visualization problems within a reasonable amount of time while also providing sophisticated visuals that are meaningful to scientists and engineers.

In this dissertation, novel techniques are presented to facilitate surface visualization. First, a particle system for mesh extraction is parallelized on the graphics processing unit (GPU) with a red-black update scheme to achieve an order of magnitude speed-up over a central processing unit (CPU) implementation. Next, extending the red-black technique to multiple materials showed inefficiencies on the GPU. Therefore, we borrow the underlying data structure from the closest point method, the closest point embedding, and the particle system solver is switched to hierarchical octree-based approach on the GPU. Third, to demonstrate that the closest point embedding is a fast, flexible data structure for surface particles, it is adapted to unsteady surface flow visualization at near-interactive speeds. Finally, the closest point embedding is a three-dimensional dense structure that does not scale well. Therefore, we introduce a closest point sparse octree that allows the closest point embedding to scale to higher resolution. Further, we demonstrate unsteady line integral convolution using the closest point method.
